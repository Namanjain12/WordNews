	\section{Methods}
\label{section:methods}
% Tao: Which classifier did you use?
% Hong Jin: SVM?

As Navigli \shortcite{Navigli09wordsense} noted that supervised approaches have performed best in WSD, we focus on integrating word embeddings in supervised approaches; in specific,
%We explored two approaches for performing supervised WSD using word embeddings. Firstly, we experimented with different methods of composing word embeddings to represent the context of a word and used this in conjunction with IMS, a state-of-the-art supervised WSD system. 
%Secondly, we explored the use of Neural Networks, which have produced state-of-the-art performance in many NLP tasks, in performing WSD. A LSTM network, a type of Recurrent Neural Network, is evaluated. 
we explore the use of word embeddings within the IMS framework. We focus our work on Continuous Bag of Words (CBOW) from Word2Vec,  Global Vectors for Word Representation (GLoVe) and Collobert \& Weston's Embeddings. The CBOW embeddings were trained over Wikipedia, while the publicly available vectors from GloVe and C\&W were used. Word2Vec provides 2 architectures for learning word embeddings, Skip-gram and CBOW. In contrast to Iacobacci \shortcite{Iacobacci2016} which focused on Skip-gram, we focused our work on CBOW.
In our first set of evaluations, we used tasks from Senseval-2, Senseval-3 and SemEval-2007 to evaluate the performance of our classifiers on monolingual WSD. We do this to first validate that our approach is a sound approach of performing WSD, showing improved or identical scores to state-of-the-art systems in most tasks. 

Similar to the work by Taghipour \shortcite{Taghipour15}, we experimented with the use of word embeddings as feature types in IMS. However, we did not just experiment using C\&W embeddings, as different word embeddings are known to vary in quality when evaluated on different tasks \cite{schnabel2015evaluation}. We performed evaluation on several tasks. For the Lexical Sample (LS) tasks of Senseval-2 \cite{senseval2-LS-kilgarriff2001} and Senseval-3 \cite{senseval3-LS-mihalcea2004}, we evaluated our system using fine-grained scoring. For the All Words (AW) tasks, fine-grained scoring is done for Senseval-2 \cite{senseval2-AW-palmer2001} and Senseval-3 \cite{senseval3-AW-snyder2004}; both the fine \cite{semeval2007-fine-pradhan2007} and coarse-grained \cite{semeval2007-coarse-navigli2007} All Words tasks in SemEval-2007 were used. In order to evaluate our features on the All Words task, we trained IMS and the different combinations of features on the One Million Sense-Tagged corpus \cite{taghipour2015one}.

To compose word vectors, one method (used as a baseline) is to sum up
the word vectors of the words in the surrounding context or
sentence. We primarily experimented on this method of composition, due
to its good performance and short training time. For this, every word
vector for every lemma in the sentence (exclusive of the target word)
was summed into a context vector, resulting in $d$ features. Stopwords
and punctuation are discarded. In Turian's
\shortcite{Turian10wordrepresentations} work, two hyperparameters ---
the capacity (number of dimensions) and size of the word embeddings
--- were tuned in his experiments. We follow his protocol and perform
the same in our experiments.

As the remaining features in IMS are binary features, they are not
comparable to the word embeddings which can have unbounded values,
leading to unbalanced influence.  As suggested by Turian
\shortcite{Turian10wordrepresentations}, we should scale down the word
embeddings values, to place their values in the same range as the
other features. The embeddings are scaled to control their standard
deviations. We implement a variant of this technique as done by
Taghipour \shortcite{Taghipour15}, in which we set the target standard
deviation for each dimension. A comparison of different values of the
scaling parameter, $\sigma$ is done. For each $i \in \{1, 2, .. d\}$:
\\

$E_{i} \leftarrow \sigma \times \frac{E_{i}}{stdev(E_{i})} $, where
$\sigma$ is a scaling constant that sets the target standard deviation
\\

We evaluate the effect of varying the scaling factor with the feature
of the sum of the surrounding word vectors. We used word embeddings of 50
dimensions.

\begin{table}[th]
	\caption{Effects of varying scaling factor on C\&W embeddings.}
	\label{table:wordembeddings-accuracy}
	\begin{center}
		\begin{tabular}{| p{7cm} | p{4cm} | p{4cm} |}
			\hline
			Method & Senseval-2 Accuracy & Senseval-3 Accuracy \\
			\hline
			C\&W, unscaled & 0.569 & 0.641 \\
			\hline
			C\&W, $\sigma _{=0.15}$ & 0.665 & 0.731 \\
			\hline
			C\&W, $\sigma _{=0.1}$ & {\bf0.672} & {\bf0.739} \\
			\hline
			C\&W, $\sigma _{=0.05}$ & 0.664 & 0.735 \\
			\hline
		\end{tabular}
	\end{center}
\end{table}

We experiment with different word embeddings and varied the value of
$\sigma$. Similar to Turian \shortcite{Turian10wordrepresentations}
and Taghipour \shortcite{Taghipour15}, we found that a value of 0.1
for $\sigma$ works well, as seen in Table
\ref{table:wordembeddings-accuracy}. The number of dimensions, known
as the capacity, of the word embeddings was tuned by Turian
\shortcite{Turian10wordrepresentations}. Hence, we varied the values
of the capacity for CBOW and GloVe.  In Tables
\ref{table:wordembeddings-word2vec-accuracy} and
\ref{table:wordembeddings-glove-accuracy} for CBOW and GloVe, the
context sum feature works optimally with 50 dimensions.

In Table \ref{table:top-LS}, we compare the performances of our system
on Senseval-2 (held in 2001) and Senseval-3's (held in 2004) Lexical
Sample tasks with IMS, the top system for each task, and other recent
systems that evaluated on the same task.

\begin{table}[th]
	\caption{Effects of varying capacity on CBOW.}
	\label{table:wordembeddings-word2vec-accuracy}
	\begin{center}
		\begin{tabular}{| p{7cm} | p{4cm} | p{4cm} |}
			\hline
			Method & Senseval-2 Accuracy & Senseval-3 Accuracy \\
			\hline
			CBOW, $dimensions_{=50}$ & {\bf0.680} & {\bf0.741} \\
			\hline
			CBOW, $dimensions_{=300}$ & 0.669 & 0.731 \\
			\hline
		\end{tabular}
	\end{center}
\end{table}

\begin{table}[th]
	\caption{Effects of varying capacity on GloVe.}
	\label{table:wordembeddings-glove-accuracy}
	\begin{center}
		\begin{tabular}{| p{7cm} | p{4cm} | p{4cm} |}
			\hline
			Method & Senseval-2 Accuracy & Senseval-3 Accuracy \\
			\hline
			GloVe, $dimensions_{=50}$ & {\bf0.678} & {\bf0.741} \\
			\hline
			GloVe, $dimensions_{=100}$ & 0.668 & 0.734 \\
			\hline
			GloVe, $dimensions_{=200}$ & 0.666 & 0.73 \\
			\hline
		\end{tabular}
	\end{center}
\end{table}

Because each dimension is a feature that is used by IMS, if there are
more dimensions, then there are more features. This may result in
overfitting on small datasets. This is a possible reason that the
smaller number of dimensions work better.

\begin{table}[th]
	\caption{Comparison of systems on Lexical Sample tasks. Rank 1 system refers to the top system for the specified task during the evaluation.}
	\label{table:top-LS}
	\begin{center}
		\begin{tabular}{| p{7cm} | p{4cm} | p{4cm} |}
			\hline
			Method & Senseval-2 LS Accuracy & Senseval-3 LS Accuracy \\
			\hline
			IMS + Context Sum, CBOW $\sigma _{=0.1}$ (proposed) & 0.680 & 0.741 \\
			\hline
            IMS + CBOW $\sigma _{=0.15}$ (proposed) & 0.670 & 0.734 \\
			\hline
			
			IMS & 0.653 & 0.726\\
			\hline
			Rank 1 System & 0.642 & 0.729 \\
			\hline
			\newcite{rothe2015autoextend} & 0.666 & 0.736 \\
			\hline
			\newcite{Taghipour15} & 0.662 & 0.734 \\
			\hline
           	IMS + Word2Vec (Skip-gram) \shortcite{Iacobacci2016}  & {\bf0.699} & {\bf0.752} \\
			\hline
			Most Frequent Sense (Baseline) & 0.476 & 0.552 \\
			\hline
		\end{tabular}
	\end{center}
\end{table}

Apart from the Lexical Sample tasks, we evaluated our systems on the All Words tasks of Senseval-2, Senseval-3 and SemEval-2007, which were evaluated on by Zhong and Ng \shortcite{Zhong2010}. As seen in Table \ref{table:All-AW}, our enhancements to IMS to make use of word embeddings give better results on All Words task than the original IMS and the respective Rank~1 systems from the original tasks. It also outperforms several recent systems developed and evaluated in recent papers. We note that although our system increased accuracy on IMS on several
% Tao: which test did you use? 
% Hong Jin: mcnemar's test
% Min: edited in
tasks, the differences were not statistically significant (as measured
using McNemar's test for paired nominal data).

\begin{table}[th]
	\caption{Accuracy of our system on Senseval-2, Senseval-3, SemEval-2007 All Words task.}
	\label{table:All-AW}
	\begin{center}
		\begin{tabular}{| p{4cm} | p{2cm} | p{3cm} | p{3cm} | p{3cm} | }
			\hline
			Method & Senseval-2 AW Accuracy & Senseval-3 AW Accuracy & SemEval-2007 Fine-Grained & SemEval-2007 Coarse-grained \\
			\hline
			IMS + CBOW, 
			
			$\sigma _{=0.1}$ (proposed) & 0.677 & 0.679 & 0.604 & 0.826  \\
			\hline
            IMS + 
			
			CBOW, $\sigma _{=0.15}$ (proposed) & 0.673 & 0.675 & {\bf0.615} & {\bf 0.828 } \\
			\hline
			
			\cite{Taghipour15} & -& {\bf0.682} & - & - \\
			\hline
			\cite{chen2014} & - & - & - & 0.826  \\
			\hline
			IMS (on One Million Sense-Tagged dataset) & 0.682 & 0.674 & 0.585 & 0.816 \\
            \hline
            IMS + Skip-gram \cite{Iacobacci2016}  & {\bf0.683} & {\bf0.682} & 0.591 & - \\
			\hline
			IMS (original) & 0.682 & 0.676 & 0.583 & 0.826   \\
			\hline
			Top System during the task & {\bf0.69} & 0.652 & 0.591 & 0.825  \\
			\hline
			WordNet Sense 1 & 0.619 & 0.624 & 0.514 & 0.789\\
			\hline
		\end{tabular}
	\end{center}
\end{table}


It can be seen in Table \ref{table:top-LS} and \ref{table:All-AW} that the simple enhancement of integrating word embedding using the baseline composition method, followed by the scaling step, improves the existing IMS system, and we get performance comparable to or better than top approaches in both Lexical Sample tasks and All Words tasks. 

\begin{table}[th]
	\caption{Accuracy of adding word embeddings to IMS on Senseval-2, Senseval-3 Lexical Sample tasks and SemEval-2007 All Words task}
	\label{table:full}
	\begin{center}
\begin{tabular}{|p{1cm}|p{0.5cm}|p{1cm}|p{1.5cm}|p{1.5cm}|p{1.5cm}|p{1.5cm}|p{1.5cm}|p{1.5cm}|p{1.5cm}|}
Type & Size & Scaling & Compose & SE-2 LS & SE-3 LS & SE-2 AW & SE-3 AW & SE-2007 Fine grained & SE-2007 Coarse grained \\
\hline
Senna&50&0.05&Sum&0.666&0.734&0.679&0.673&0.594&0.818 \\
\hline
Senna&50&0.1&Sum&0.671&0.738&0.678&0.673&0.6&0.819 \\
\hline
Senna&50&0.15&Sum&0.666&0.732&0.675&0.672&0.598&0.817 \\
\hline
CBOW&50&0.05&Sum&0.672&{\bf 0.744}&{\bf 0.68}&0.677&0.604&0.824\\
\hline
CBOW&50&0.1&Sum&{\bf 0.68}&0.741&0.677&0.679 &0.604 & 0.826\\
\hline
CBOW&50&0.15&Sum&0.67&0.734&0.673&0.675&{\bf 0.615}&{\bf 0.828}\\
\hline
Glove&50&0.05&Sum&0.675&0.738&0.676&0.678&0.596&0.819 \\
\hline
Glove&50&0.1&Sum&0.679&0.741&0.678&0.68&0.594&0.819 \\
\hline
Glove&50&0.15&Sum&0.674&0.731&{\bf 0.68}&0.678&0.591&0.819 \\
\hline
CBOW&200&0.05&Sum&0.679&0.742&0.679&0.68&0.602&0.823 \\
\hline
CBOW&200&0.1&Sum&0.669&0.731&0.676&0.675&0.602&0.82 \\
\hline
CBOW&200&0.15&Sum&0.651&0.715&0.667&0.673&0.594&0.822 \\
\hline
Glove&200&0.05&Sum&0.682&0.741&0.68&{\bf0.682}&0.6&0.823 \\
\hline
Glove&200&0.1&Sum&0.666&0.73&0.677&0.679&0.591&0.827 \\
\hline
Glove&200&0.15&Sum&0.654&0.706&0.674&0.675&0.591&0.826 \\
\hline
Senna&50&0.1&Concat&0.659&0.724&0.679&0.674&0.585&0.818 \\
\hline
CBOW&50&0.1&Concat&0.66&0.725&0.678&0.672&0.581&0.816\\
\hline
CBOW&200&0.1&Concat&0.667&0.729&0.675&0.67&0.591&0.819\\
\hline
Glove&50&0.1&Concat&0.657&0.722&0.679&0.671&0.583&0.818\\
\hline
Glove&200&0.1&Concat&0.664&0.728&0.677&0.669&0.587&0.817\\
\hline
\end{tabular}
	\end{center}
\end{table}

We note that a smaller size of the word embeddings generally improved
performance on the Lexical Sample task, however, this effect was not
observed in the All Words task. We also note that relatively poorer
performance in the Lexical Sample tasks may not necessarily result in
poor performance on the All Words task. We see from the results that
the combination of \cite{Taghipour15}'s scaling strategy and summation
produced results better than the proposal in \cite{Iacobacci2016} to
concatenate and average (0.651 and 0.654), suggesting that the scaling
factor is important for the integration of word embeddings for
supervised WSD.

%\iffalse
\subsection{LSTM Network}

A Long Short Term Memory (LSTM) network is a type of Recurrent Neural
Network which has recently been shown to have good performance on many
NLP classification tasks. Unlike normal neural networks which can only
accept a fixed size vector as an input, recurrent neural networks
accept variable sized inputs. As such, recurrent neural networks can
operate over sequences of word vectors and perform operations on them
sequentially. The potential benefit of this approach over our existing
approach in IMS is this is that the neural network is able to use
information about the sequence of words in classification. Examples of
using a neural network for classification are
\cite{socher2011parsing} and \cite{socher2013recursive}.  
% Min: how well did they do?  Comparable to yours?
Yuan \shortcite{yuan2016word} explored the use of LSTM networks, using
label propagation, for WSD. In our approach, we explore a simpler
na\"{\i}ve approach instead without the use of label propagation.

For the Lexical Sample tasks, we train the model on the training data
provided for the task. For the All Words task, we trained the model on
the One Million Sense-Tagged dataset. For each task, similar to IMS,
we train a model for each word.

One problem we encountered with this approach was that there were very
few training examples per sense. Although as a whole, the dataset had
many training examples, as we needed to train a model for each word,
this resulted in a small number of relevant training instances for a
single word.

\begin{table}[th]
	\caption{Accuracy of a basic LSTM approach on the Lexical
          Sample tasks.}
	\label{table:NN-LS}
	\begin{center}
		\begin{tabular}{| p{6cm} | p{4cm} | p{4cm} |}
			\hline
			Method & Senseval-2 Accuracy & Senseval-3 Accuracy \\
			\hline
			LSTM approach (Proposed) & 0.458  & 0.603 \\
			\hline
			IMS & 0.653 & {\bf0.726}\\
			\hline
			Rank 1 System during the task & {\bf0.642} & 0.729 \\
			\hline
			Most Frequent Sense (Baseline) & 0.476 & 0.552 \\
			\hline
		\end{tabular}
	\end{center}
\end{table}

\begin{table}[th]
	\caption{Accuracy of a basic LSTM approach on the All Words tasks.}
	\label{table:NN_AW}
	\begin{center}
		\begin{tabular}{| p{7cm} | p{2cm} | p{2cm} | p{2cm} | }
			\hline
			Method & Senseval-2 Accuracy & Senseval-3 Accuracy\\
			\hline
			LSTM approach (Proposed) & 0.619  & 0.623  \\
			
			\hline
			IMS (trained on One Million Sense-Tagged dataset) & 0.682 & {\bf0.674} \\
			\hline
			Rank 1 System during the task & {\bf0.69} & 0.652  \\
			\hline
			Wordnet Sense 1 & 0.619 & 0.624  \\
			\hline
		\end{tabular}
	\end{center}
\end{table}

The performance of the LSTM is poor in both tasks (Tables
\ref{table:NN-LS} and \ref{table:NN_AW}). 
% Min: Repetitive -- should have some examples to show this for an
% individual word or two, or to show that words that have more
% instances have markedly better performance.
The models converge to just
using the most common sense. The reason for this is that WSD suffers
from the problem of data sparsity. Although there are many training
instances in total, the average amount of training examples for each
individual sense is low.

\section{Introduction}
Learning a new language from language learning websites is 
time consuming. Research shows that regular practice, guessing, 
memorization~\cite{rubin75} as well as immersion into real 
scenarios~\cite{naiman78} hastens language learning process. 
To make second language learning 
attractive and efficient, we seek to interleave language learning 
with a popular daily activity: online news reading.
% Min: not needed. 
%% Further recent increase in the popularity of 
%% portable devices and computers has made online news reading 
%% popular~\cite{yarlh2012}. We leverage on this culture to provide 
%% users of news websites with an opportunity to learn a second language.

%% It is observed that language development in children begins with 
%% vocabulary acquisition before syntax and other aspects are learnt~\cite{}BUG. 
%% One may extrapolate by assumption that vocabulary acquisition is perhaps 
%% the first step in language learning, in general. Research~\cite{}BUG also 
%% shows that extensive reading builds up vocabulary. This, however, could 
%% make learning time consuming and boring, if the learner is uninterested 
%% in the reading material. We, therefore, seek to enable learners build 
%% their vocabulary efficiently with an enjoyable user experience.

%Existing language learning software  
%teach through structured lessons 
%or through strategies for memorizing the new vocabularies. 
%In the first category, 
%learning in lessons, they manually design some lessons to help their 
%lessons are purposefully designed to help users easily learn a foreign language.
%For the second category users are guided to recite lists of words, or provided with 
%a translation for their input word in the foreign language.

Most existing language learning software are either instruction-driven
or user-driven.  Duolingo\footnote{\url{https://www.duolingo.com/}} is
a popular instruction-driven system that teaches through structured
lessons.
% Tao: I think users are able to select lessons by difficulty and thus match their skills.
% are difficult to customise according to the learner's skill level. 
% Muthu: agreed. 
Instruction driven systems demand dedicated learner time on a daily
basis and are limited by learning materials as lesson curation is
often labor-intensive.

In contrast, many people informally use Google
Translate\footnote{\url{https://translate.google.com/}} to learn
vocabulary, making it a prominent example of a user-driven system.
% Tao: It is better to  postpone mentioning the specific two languages. Perhpas mention it 
% at the end of this paragraph. Too many Chinese (English) makes reading a little bit hard. 
% In addition, our current system is to help a English speaker to learn Chinese, not suitable 
% for a Chinese speaker to learn English.
% Muthu: okay. I thought, I saw screenshots for both ways.
% Anway, we will stick with Eng -> Chinese throughout.
Translate, however, lacks the rigor of a learning platform as it lacks
tests to allow learners to demonstrate mastery.  In our work, we merge
learning and assessment within the single activity of news reading.
%Learners are tested while they learn. 
Our system also adapts to the learner's skill during assessment.
%~\footnote{\url{https://neltachoutari.wordpress.com/tag/%vocabulary/}}. 
%Perhaps, this justifies the poor adoption of current second language learning 
%systems. 

We propose a system to enable online news readers to efficiently learn
a new language.  Our prototype targets Chinese language learning while
reading English language news. Learners are provided translations of
open-domain words for learning from an English news page. In the same
environment -- for words that the system deems mastered by the learner
-- learners are assessed by replacing the original English text in the
article with their Chinese translations and asked to translate them
back given a choice of possible translations.  The system, deployed as
a Chrome web browser extension, is triggered when readers visit a
preconfigured list of news websites.

A key design property of our language learning extension is only
active on certain news websites.  This is important as news articles
typically are classified with respect to a news category, such as {\it
  finance}, {\it world news}, and {\it sports}.  If we know which
category of news the learner is viewing, we can leverage this
contextual knowledge to improve the learning experience.

In the development of the system, we discovered two key components
that can be affected by this context modeling.  We report on these
developments here.  In specific, we propose algorithms: \textbf{(i)}
for translating English words to Chinese from news articles,
\textbf{(ii)} for generating distractor translations for learner
assessment.
%The system even adapts to the user's current knowledge level during assessment.
%The service is available as desktop / mobile / web software including a chrome extension. 
%We mainly compare our system  with the aforementioned two software.
%Table~\ref{table:difference_summary} summarises important differences between 
%our system and all these existing tools. Each difference serves as a motivation 
%for developing our extension.
%``Duolingo is a free language-learning and crowdsourced text translation 
%platform''\footnote{\url{http://en.wikipedia.org/wiki/Duolingo}}.
%Most people start to use Duolingo when they know a little or nothing about 
%the new language. They starting from some basic lessons and improve step by step.
%However, our target audience is a mix novice and intermediate level learners of the foreign language. 
%We can not only help beginners learn 
%a new language but also help them continue their learning by allowing them to practice 
%their foreign language. There are also a lot articles with their translations in 
%Duolingo, but all the articles and their translations are manually added by 
%Duolingo or users from Duolingo. Therefore, parallel articles in Duolingo are old and 
%limited. However, our chrome extension is always working even for those up to the 
%minute news and our user can just practice their foreign language in their daily 
%readings.

%\textbf{Google Translate:} ``Highlight or right-click on a %section of text and click
%on Translate icon next to it to translate it to your 
%language''\footnote{\url{http://en.wikipedia.org/wiki/%Google_Chrome_Extensions}}. 
%% Tao: please cite
%Google Translate is a chrome extension that displays only %the translation when user 
%select a section, which can be a word, a phrase, a sentence %or even a whole page. 
%Our chrome extension will translate a single word only, and %display the translation,
%following with the pronunciations and example sentences to %help user understand and 
%remember this word. Compared with our extension, Google %Translate is more like an extension 
%to help user understand the content of the page. %Furthermore, our extension will display 
%the most appropriate translation as it will refer to the %context of the word.

%\begin{table}[ht]
%  \caption{Summary of the differences}
%  \label{table:difference_summary}
%  \begin{center}
%  \begin{tabular}{| p{2.4cm} | p{1.2cm} | p{1.2cm} |  p{1.2cm} |}
%    \hline
%    & Duolingo & Google Translate & Chrome Extension \\
%    \hline
%    Lessons & Yes & No & No \\
%    \hline
%    User's foreign language level & Low & Low-High & Low-High \\
%    \hline
%    Time consuming & Yes & No & No\\
%    \hline
%    Resource & Limited & Infinite & Infinite \\
%    \hline
%    Customizable & Yes & No & Yes \\
%    \hline
%    Link to External Dictionary & No & No & Yes \\
%    \hline
%  \end{tabular}
%  \end{center}
%\end{table}

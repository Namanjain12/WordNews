\section{Introduction}
People read news every day. With the increasing popularity of 
portable devices and computers, more and more people are reading 
news from news websites \cite{yarlh2012}.
\\
Weleverage on this culture to provide users of news websites 
with an opportunity to learn a second language.
Learning a new language from language learning websites is very time consuming. 
To this end, we propose a system to enable visitors to news websites to 
efficiently learn a new language while they are reading news. We propose to 
build the system as a browser extension which would run on the client side 
when readers visit news websites from preconfigured list.
We choose to build the extension for Google Chrome over others popular browsers 
such as Firefox, Safari since the former has seen much growth recently (Leather, 2014). 
Further,Internet Explorer, the most widely used browser, doesn't support extensions.
\\
We propose to solve two problems simultaneously: algorithmic  
teaching of a second language and improving efficiency of users in 
memorizing a vocabulary.
Furthermore, as our Chrome Extension needs to modify the original page, 
minimizing the quantum of code to inject to the original source would 
improve maintainability and future extensions of the system 
for deployment in other types of websites.
\\
Learning a new vocabulary is the most time consuming and boring part of 
language learning\footnote{\url{https://neltachoutari.wordpress.com/tag/vocabulary/}}. Perhaps, this justifies the 
poor adoption of current second language learning systems. We, therefore, 
focus on enabling language learners build their vocabulary efficiently while 
providing them with an enjoyable user experience.
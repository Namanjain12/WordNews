\section{Introduction}
Learning a new language from language learning websites is 
time consuming. Therefore, to make second language learning 
attractive and efficient we seek to interleave language leaning 
with a popular daily activity, online news reading. Research shows 
that regular practice, guessing, memorization~\cite{rubin91} as well 
as immersion into a real context that requires usage~\cite{naiman78} hastens 
the learning process. Further recent increase in the popularity of 
portable devices has made online news reading popular than ever 
before~\cite{yarlh2012}. We leverage on this culture to provide 
users of news websites with an opportunity to learn a second 
language.

We propose a system to enable online news readers to efficiently learn 
a new language, Chinese, while they are reading news online. The system, deployed 
as a Chrome extension~\footnote{a software extension to the Chrome web browser}  
is triggered when readers visit news websites on a preconfigured list.

Since learning a new vocabulary is the most time consuming and boring part of 
language learning~\footnote{\url{https://neltachoutari.wordpress.com/tag/vocabulary/}}. 
Perhaps, this justifies the poor adoption of current second language learning 
systems. We, therefore, focus on enabling language learners build their vocabulary 
efficiently while providing them with an enjoyable user experience.

Many existing language learning software teach through structured lessons 
or through strategies for memorizing the new vocabularies. 
%In the first category, 
%learning in lessons, they manually design some lessons to help their 
%lessons are purposefully designed to help users easily learn a foreign language.
%For the second category users are guided to recite lists of words, or provided with 
%a translation for their input word in the foreign language. 
While Duolingo\footnote{\url{https://www.duolingo.com/}} teaches through lessons, 
Google Translate \footnote{\url{https://translate.google.com/}} allows you to learn by 
translating your inputs. 
%The service is available as desktop / mobile / web software including a chrome extension. 
%We mainly compare our system  with the aforementioned two software.
%Table~\ref{table:difference_summary} summarises important differences between 
%our system and all these existing tools. Each difference serves as a motivation 
%for developing our extension.
%``Duolingo is a free language-learning and crowdsourced text translation 
%platform''\footnote{\url{http://en.wikipedia.org/wiki/Duolingo}}.
%Most people start to use Duolingo when they know a little or nothing about 
%the new language. They starting from some basic lessons and improve step by step.
%However, our target audience is a mix novice and intermediate level learners of the foreign language. 
%We can not only help beginners learn 
%a new language but also help them continue their learning by allowing them to practice 
%their foreign language. There are also a lot articles with their translations in 
%Duolingo, but all the articles and their translations are manually added by 
%Duolingo or users from Duolingo. Therefore, parallel articles in Duolingo are old and 
%limited. However, our chrome extension is always working even for those up to the 
%minute news and our user can just practice their foreign language in their daily 
%readings.

\textbf{Google Translate:} ``Highlight or right-click on a section of text and click
on Translate icon next to it to translate it to your 
language''\footnote{\url{http://en.wikipedia.org/wiki/Google_Chrome_Extensions}}. 
% Tao: please cite
Google Translate is a chrome extension that displays only the translation when user 
select a section, which can be a word, a phrase, a sentence or even a whole page. 
Our chrome extension will translate a single word only, and display the translation,
following with the pronunciations and example sentences to help user understand and 
remember this word. Compared with our extension, Google Translate is more like an extension 
to help user understand the content of the page. Furthermore, our extension will display 
the most appropriate translation as it will refer to the context of the word.

%\begin{table}[ht]
%  \caption{Summary of the differences}
%  \label{table:difference_summary}
%  \begin{center}
%  \begin{tabular}{| p{2.4cm} | p{1.2cm} | p{1.2cm} |  p{1.2cm} |}
%    \hline
%    & Duolingo & Google Translate & Chrome Extension \\
%    \hline
%    Lessons & Yes & No & No \\
%    \hline
%    User's foreign language level & Low & Low-High & Low-High \\
%    \hline
%    Time consuming & Yes & No & No\\
%    \hline
%    Resource & Limited & Infinite & Infinite \\
%    \hline
%    Customizable & Yes & No & Yes \\
%    \hline
%    Link to External Dictionary & No & No & Yes \\
%    \hline
%  \end{tabular}
%  \end{center}
%\end{table}
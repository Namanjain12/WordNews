\section{Related Work}
% Min: don't need this header -- obvious
% \subsection{Word Sense Disambiguation}

Word Sense Disambiguation is a well-studied problem, in which many
methods have been applied. Existing methods can be broadly categorised
into supervised approaches, where machine learning techniques are used
to learn from labeled training data; and unsupervised knowledge-rich
techniques, which do not rely on labeled data. Unsupervised techniques
are knowledge-rich, and rely heavily on knowledge bases and thesaurus,
such as WordNet. It is noted by Navigli \shortcite{Navigli09wordsense}
that supervised approaches using memory-based learning and SVM
approaches have worked best.
%For these approaches, it is common that the only knowledge used is the first sense in WordNet, which is used as a fallback if the system is unable to disambiguate the word in the test data. 

Supervised approaches involve the extraction of features and then
classification using machine learning. Zhong and Ng
\shortcite{Zhong2010} developed an open-source WSD system, {\it
  ItMakesSense} (hereafter, IMS), which was considered the
state-of-the-art at the time it was developed.  It is a supervised WSD
system, which had to be trained prior to use. IMS uses three feature
types, 1. individual words in the context surrounding the target word,
2. specific ordered sequences of words appearing at specified offsets
from the target word, 3. Part-Of-Speech tags of the surrounding 3
words.

Each of the features are binary features, and IMS trains a model for
each word. IMS then uses an support vector machine (SVM) for
classification. IMS is open-source, provides state-of-the-art
performance, and is easy to extend. As such, our work features IMS and
extends off of this backbone.

Training data is required to train IMS.  We make use of the
One-Million Sense-Tagged Instances \cite{taghipour2015one} dataset,
which is the largest dataset we know of for training WSD systems, in
training our systems for the All Words tasks.

WSD systems can be evaluated using either fine-grained scoring or
coarse-grained scoring. Under fine-grained scoring, every sense is
equally distinct from each other, and answers must exactly match. 
% Min: need to introduce wordnet somewhere.
% Hong Jin: added context that wordnet is the sense inventory
WordNet is often used as the sense inventory for monolingual WSD tasks. However, WordNet is a fine-grained resource, and even human annotators have
trouble distinguishing between different senses of a word
\cite{edmonds2002introduction}.  In contrast, under coarse-grained
scoring, similar senses are grouped and treated as a single sense.  In
some WSD tasks in SemEval, coarse-grained scoring was done in order
to deal with the problem of reliably distinguishing fine-grained
senses. 
% Hong Jin: removed because repetitive
%We note that WSD is somewhat subjective, in that the grouping
%and granularity of senses are debatable and may not be well-defined.
%% Min: not yet clear where this goes
% An important measure for any classification task is the
% inter-annotator agreement. The inter-annotator agreement is considered
% the upper bound of a task.

\subsection{Cross-Lingual Word Sense Disambiguation}
Cross-Lingual WSD was, in part, conceived as a further attempt to
solve this issue. In Cross-Lingual WSD, the specificity of a sense is
determined by its correct translation in another language. The sense
inventory is the possible translations of each word in another
language. Two instances are said to have the same sense if they map to
the same translation in that language.
SemEval-2010~\cite{Lefever2010}\footnote{\url{http://stel.ub.edu/semeval2010-coref/}}
and SemEval-2013~\cite{Lefever2013}\footnote{\url{https://www.cs.york.ac.uk/semeval-2013/}}
featured iterations of this task. These tasks featured
English nouns as the source words, and word senses as translations in
Dutch, French, Italian, Spanish and German.

Traditional WSD approaches are used in Cross-Lingual WSD, although
some approaches leverage statistical machine translation (SMT) methods
and features from translation. Cross-Lingual WSD involves training by
making use of parallel or multilingual corpora. In the Cross-Lingual
WSD task in SemEval-2013, the top performing approaches used either
classification or SMT approaches.

\subsection{WSD with Word Embeddings}

In NLP, words can be represented 
%in a vector space
%model. 
%Traditionally, this has been done with {\it one-hot} binary
%vectors, where there is only one non-zero value in a high-dimensional
%vector. In this encoding, each dimension represents the presence of a
%word, and the number of dimensions of the vector space is the size of
%the vocabulary. In one-hot encoding, all words are considered to be
%independent of each other; there is no notion of word similarity and
%all words are independent of each other.  A key weakness with the
%one-hot representation is that the large number of dimensions makes
%machine learning prone to overfitting.  
with a distributed representation, such as word embeddings, which encodes
words into a low dimensional space. In word embeddings, information
about a word is distributed across multiple dimensions, and similar
words are expected to be close to each other in the vector space. Examples of word
embeddings are Continuous Bag of Words \cite{mikolovword2vec},
Collobert \& Weston's Embeddings \cite{collobert2008unified}, and
GLoVe \cite{pennington2014glove}. We implemented and evaluated the use
of word embedding features using these three embeddings in IMS.

% An unsupervised approach using word embeddings for WSD is described by
% Chen \shortcite{chen2014}. This uses a model for finding
% representation of senses, rather than just for words, initialized
% using WordNet's glosses of senses. These sense vectors can then be
% used during Word Sense Disambiguation. A context vector can be
% computed by taking the average of the words in a sentence. For
% disambiguating a single word, the sense with the sense vector that
% gives maximum Cosine Similarity with this context vector is chosen as
% the result for disambiguation. Chen {\it et al.} gives an algorithm to
% disambiguate words starting from the words with fewer senses first.

An unsupervised approach using word embeddings for WSD is described by Chen \shortcite{chen2014}. This approach finds representation of senses, instead of words, and computes a context vector which is used during disambiguation. 

A different approach is to work on extending existing WSD
systems. Turian \shortcite{Turian10wordrepresentations} suggests that
for any existing supervised NLP system, a general way of improving
accuracy would be to use unsupervised word representations as
additional features. Taghipour \shortcite{Taghipour15} used C\&W
embeddings as a starting point and implemented word embeddings as a
feature type in IMS. For a specified window, vectors for the
surrounding words in the windows, excluding the target word, are
obtained from the embeddings and are concatenated, producing $d *
(w-1)$ features, where $d$ is the number of dimensions of the vector,
and $w$ is the window size. Each feature is a floating point number,
which is the value of the vector in a dimension. We note that
\cite{Taghipour15} only reported results for C\&W embeddings, and did
not experiment on other types of word embeddings.

Other supervised approaches using word embeddings include AutoExtend
\cite{rothe2015autoextend}, which extended word embeddings to create
embeddings for synsets and lexemes. In their work, they also extended
IMS, but used their own embeddings. The feature types
introduced by this work bear similarities to how Taghipour used
word embeddings, but without Taghipour's method of scaling each
dimension of the word embeddings.

% Min: BUG fix this into prose if needed. 
% Hong Jin: not needed
% Summary:
%  - Not many comparison works: \cite{Iacobacci2016}

To conclude, word embeddings have been used in several methods to
improve on state-of-the-art results in WSD. However, to date, there
has been little work investigating how different word embeddings and
parameters affect performance of baseline methods of WSD. As far as we
know, there has only been one paper comparing the different word
embeddings with the use of basic composition methods in WSD. Iacobacci
\shortcite{Iacobacci2016} performed an evaluation study of different
parameters when enhancing an existing supervised WSD system with word
embeddings. Iacobacci noted that the integration of Word2Vec
% Min: skip grams are different from the CBOW
(skip-gram) with IMS was consistently helpful and provided the best
performance. Iacobacci also noted that the composition methods of
average and concatenation produced small gains relative to the other
composition strategies introduced. However, Iacobacci did not
investigate the use of \cite{Taghipour15}'s scaling strategy, which
was crucial to improve the performance of IMS.

We also did not find any recent work attempting to integrate modern
WSD systems for real-world education usage, and to evaluate the WSD
system based on the requirements and suitability for education use.
We aim to fill this gap in applied WSD with this work.

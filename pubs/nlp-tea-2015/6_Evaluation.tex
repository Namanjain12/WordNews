\section{Platform Viability and Usability Survey}

We have thus far described and evaluated two critical components that
can benefit from capturing the learner's news article context.  In the
larger context, we also need to check the viability of second language
learning intertwined with news reading.  In a requirements survey
prior to the prototype development, two-thirds of the respondents
indicated that although they have interest in learning a second
language, they only have only used language learning software
infrequently (less than once per week) yet frequently read news,
giving us motivation for our development.

Post-prototype, we conducted a summative survey to assess whether our
prototype product satisfied the target niche, in terms of interest,
usability and possible interference with normal reading activities.
We gathered 16 respondents, 15 of which were between the ages of
18--24.  11 (the majority) also claimed native Chinese language
proficiency.  

% Min: Yue, need your input here.  Something like the below.
The respondents felt that the extension platform was a viable language
learning platform (3.4 of 5; on a scale of 1 ``disagreement'' to 
5 ``agreement'') and that they would like to try it when available
for their language pair (3 of 5).

In our original prototype, we replaced the original English word with
the Chinese translation.  While most felt that replacing the original
English with the Chinese translation would not hamper their reading,
they still felt a bit uncomfortable (3.7 of 5).  This finding prompted
us to change the default learning tooltip behavior to underlining to
hint at the tooltip presence.

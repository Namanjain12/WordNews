%
%% Based on the style files for EACL 2006 by 
%%e.agirre@ehu.es or Sergi.Balari@uab.es
%% and that of ACL 08 by Joakim Nivre and Noah Smith

\documentclass[11pt]{article}
\usepackage{coling2016}
\usepackage{times}
\usepackage{url}
\usepackage{latexsym}
\usepackage{CJK}
\usepackage{multirow}
\AtBeginDvi{\input{zhwinfonts}}

% 8 pages + 2 pages of references
%\setlength\titlebox{5cm}

% You can expand the titlebox if you need extra space
% to show all the authors. Please do not make the titlebox
% smaller than 5cm (the original size); we will check this
% in the camera-ready version and ask you to change it back.

\title{A Comparison of Word Embeddings for English and Cross-Lingual Chinese Word Sense Disambiguation}
%Muthu: changed title to title case from upper case
%\title{Word Sense Disambiguation Using Word Embeddings}
%Muthu: Isn't this a more accurate title?
%\title{Cross-lingual Word Sense Disambiguation using Word Embeddings}
%\title{A Simplified Approach to Augmenting Word Embeddings to Word Sense Disambiguation}




% \author{
% Hong Jin Kang$^{1}$, Tao Chen$^{2}$,  Muthu Kumar Chandrasekaran$^{1}$, Min-Yen Kan$^{1,2}$ \\
% $^{1}$School of Computing, National University of Singapore\\
% $^{2}$NUS Interactive and Digital Media Institute\\
% {\tt kanghongjin@gmail.com }\\
% {\tt  \{taochen,muthu.chandra,kanmy\}@comp.nus.edu.sg} \\
% }

\date{June 2016}

% Tao: As the recent ACL paper~\cite{Iacobacci2016} has evaluated word embeddings for WSD, we have to change our statement, e.g., it is not proper to say "no comparison".


\begin{document}
\maketitle
%\begin{abstract}
 % A key feature of the language-learning application, {\it WordNews}, is to 
 % translate English words to Chinese. The task of Word Sense Disambiguation 
 % (WSD) is the task of identifying the meaning of words in context. We treat 
 % the translation task of WordNews as Cross-Lingual Word Sense Disambiguation, 
 % a variant of WSD.
  
%  To perform Cross-Lingual WSD, we experiment with semi-supervised approaches using word embeddings. 
  % Tao: is IMS state-of-the-art?
  % Hong Jin: it was state-of-the-art just a couple of years ago. Removed mentions of IMS being state-of-the-art
 % We modify an existing WSD system, 
%Muthu: full exopansion when you mention it the first time. 
%`It Makes Sense' (IMS), 
%to make use of word embeddings. 
%We evaluate our approach on the Lexical Sample and All Words tasks in SemEval-2007, Senseval-2, and Senseval-3. We found that word embeddings improves the performance of the existing WSD system. 
%  To evaluate our chosen WSD system on Cross-Lingual WSD, we constructed a publicly available human-annotated English-Chinese evaluation dataset from
%Muthu: is there an unreal variant of new articles?
% Hong JIn: removed
% news articles, and evaluated our system on it.
%Muthu: Did you evaluate the integrated system. If so, mention that result 
%also here
% Hong Jin: mentioned that we evaluated on the CLWSD dataset. I did not carry out an evaluation for the version of WordNews using IMS for WS D
% Finally, we integrated the system into a fork of WordNews.
%\end{abstract}

\begin{abstract}
Word embeddings are now ubiquitous forms of word representation in
natural language processing.  There have been applications of 
%Word Embeddings  
%Muthu: removing title case from 'Word Embeddings' 
word embeddings for monolingual word sense disambiguation (WSD) in English,
but few comparisons have been done.  This paper attempts to bridge
that gap by examining popular embeddings for the task of monolingual
English WSD.  Our simplified method leads to comparable
state-of-the-art performance without expensive retraining.

Cross-Lingual WSD -- where the word senses of a word in a source
language $e$ come from a separate target translation language $f$ --
can also assist in language learning; for example, when providing
translations of target vocabulary for learners.  Thus we have also
applied word embeddings to the novel task of cross-lingual WSD for
Chinese and provide a public dataset for further benchmarking.
%We have also experimented Long Short-term memory neural networks with the same embeddings.
We have also experimented with using word embeddings for LSTM networks
and found surprisingly that a basic LSTM network does not work well.
We discuss the ramifications of this outcome.
\end{abstract}

% Tao: Need to highlight the research gap (e.g., the drawbacks of existing methods for CWSD) and contributions of this work
\section{Introduction}
\label{intro}

%
% The following footnote without marker is needed for the camera-ready
% version of the paper.
% Comment out the instructions (first text) and uncomment the 8 lines
% under "final paper" for your variant of English.
% 
\iffalse
\blfootnote{
    %
    % for review submission
    %
    \hspace{-0.65cm}  % space normally used by the marker
    Place licence statement here for the camera-ready version, see
    Section~\ref{licence} of the instructions for preparing a
    manuscript.
    %
    % % final paper: en-uk version (to license, a licence)
    %
    % \hspace{-0.65cm}  % space normally used by the marker
    % This work is licensed under a Creative Commons 
    % Attribution 4.0 International Licence.
    % Licence details:
    % \url{http://creativecommons.org/licenses/by/4.0/}
    % 
    % % final paper: en-us version (to licence, a license)
    %
    % \hspace{-0.65cm}  % space normally used by the marker
    % This work is licenced under a Creative Commons 
    % Attribution 4.0 International License.
    % License details:
    % \url{http://creativecommons.org/licenses/by/4.0/}
}
\fi

%Formally learning a new language is time-consuming and requires learners to invest a significant
%amount of effort. A Chrome Extension, {\it WordNews}, was developed by \cite{tao2014} to allow users to pick up Chinese vocabulary while reading online news articles. WordNews makes language learning efficient and attractive by interleaving language
%learning with the daily activity of online news reading. 
%WordNews allows users to learn from real-world examples, and to learn words in context, which is required for effective learning of vocabulary \cite{Hirsch03readingcomprehension}.

% Tao: The logic of this section does flow well: 1) some paragraphs have duplicate information, and 2) the motivation and contribution of this paper is not clear. I prefer to move the literature review to a separate section since it is quite long (almost 1.5 pages).
% Tao: tried to edit (partial)

A word takes on different meanings, largely dependent on the context
in which it is used. For example, the word ``bank'' could mean ``slope
beside a body of water'', or a ``depository financial
institution''~\footnote{\url{http://wordnetweb.princeton.edu/perl/webwn?s=bank}}. Word
Sense Disambiguation (WSD) is the task of identifying the contextually
appropriate meaning of the word. WSD is often considered a
classification task, in which the classifier predicts the sense from a
possible set of senses, known as a sense inventory, given the target
word and the contextual information of the target word. Existing WSD
systems can be categorised into either data-driven supervised or
knowledge-rich approaches. Both approaches are considered to be
complementary to each other.

%The task of Word Sense Disambiguation (WSD) is the task of identifying the correct sense/meaning of a word out of possible senses defined in a sense inventory. 
Word embeddings have become a popular word representation formalism,
and many tasks can be done using word embeddings. The effectiveness of
using word embeddings has been shown in
% Tao: cite a few more papers
several NLP tasks \cite{Turian10wordrepresentations}. The goal of our
work is to apply and comprehensively compare different uses of word
embeddings, solely with respect to WSD. We perform evaluation of the
effectiveness of word embeddings on monolingual WSD tasks from
Senseval-2 (held in 2001), Senseval-3 (held in 2004), and
SemEval-2007. After which, we evaluate our approach on English--Chinese
Cross-Lingual WSD using a dataset that we constructed for %the use of
evaluating our approach on the translation task used in educational
applications for language learning. %, such as {\it  MindTheWord}{\footnote{\url{https://chrome.google.com/webstore/detail/mindtheword/fabjlaokbhaoehejcoblhahcekmogbom?hl=en}}} and {\it WordNews}~\cite{tao2014}.

%% Min: Nav too ordinary, try without it.
% The structure of this paper will be as follows: we have first reviewed related work and methods regarding WSD, next we will discuss the applications of WSD to a category of educational applications, and outlining possible future work finally in the conclusion. 

%The task of Word Sense Disambiguation (WSD) is the task of identifying the correct sense/meaning of a word out of possible senses defined in a sense inventory. Word embeddings is a popular technique in NLP in recent years, and many tasks can be done using 
%word embeddings. The effectiveness of
%using word embeddings has been shown in 
%% Tao: cite a few more papers
%several NLP tasks \cite{Turian10wordrepresentations}
%The goal of this work is to apply and compare different uses of word embeddings for WSD. We perform evaluation of the effectiveness of word embeddings on monolingual WSD tasks from Senseval-2 (held in 2001), Senseval-3 (held in 2004), and SemEval-2007. After which, we evaluate our approach on 
%English-Chinese Cross-Lingual WSD using a dataset that we constructed for the use of evaluating our approach on the translation task used in educational applications for language learning, such as MindTheWord {\footnote{\url{https://chrome.google.com/webstore/detail/mindtheword/fabjlaokbhaoehejcoblhahcekmogbom?hl=en}}} and WordNews.
%
%
%
%A word can have different meanings depending on the context in which it is used. For example, the word ``bank" could mean ``slope beside a body of water", or a ``depository financial institution"~\footnote{\url{http://wordnetweb.princeton.edu/perl/webwn?s=bank}}. Word Sense Disambiguation is the task of identifying the contextually appropriate meaning of the word. Word Sense Disambiguation can be considered a classification task, in which the classifier predicts the sense from a possible set of senses, known as a sense inventory, given the target word and the contextual information of the target word. Existing WSD systems can be categorised into either supervised or knowledge-rich approaches. Both approaches are considered to be complementary to each other. 
%
%
%Word Sense Disambiguation is a well-studied problem and there are many different methods. Existing methods can be broadly categorised into supervised approaches, where machine learning techniques are used to learn from labeled training data, and unsupervised knowledge-rich techniques, which do not rely on labeled data. Unsupervised techniques are knowledge-rich, and rely heavily on knowledge bases and thesaurus, such as WordNet. It is noted by Navigli \shortcite{Navigli09wordsense} that supervised approaches using memory-based learning and SVM approaches have worked best. 
%%For these approaches, it is common that the only knowledge used is the first sense in WordNet, which is used as a fallback if the system is unable to disambiguate the word in the test data. 
%
%Supervised approaches involve the extraction of features and then classification using machine learning. \shortcite{Zhong2010} developed an open-source WSD system, IMS, which was state-of-the-art at the time it was developed. It is a supervised-learning based WSD system, which first has to be trained using a set of training data. IMS uses three feature types, 1. individual words in the context surrounding the target word, 2. specific ordered sequences of words appearing at specified offsets from the target word, 3. Part-Of-Speech tags of the surrounding 3 words.
%
%% \begin{itemize}
%
%% 	\item  Surrounding Words\\
%% 	Surrounding words include individual words in the surrounding context. Sentence boundaries can be crossed in this feature. Stopwords, punctuation, character symbols, and numbers are discarded. 
%
%% 	\item Local Collocations\\
%% 	A collocation is an ordered sequence of words appearing in a specified offset from the target word. 11 location collocation features are used. They are $C_{-2},_{-2}$, $C_{-1},_{-1}$,
%% 	$C_{1},_{1}$, $C_{2},_{2}$, $C_{-2},_{-1}$, $C_{-1},_{1}$, $C_{1},_{2}$, $C_{-3},_{-1}$,
%% 	$C_{-2},_{1}$, $C_{-1},_{2}$, and $C_{1},_{3}$. $C_{i},_{j}$ refers to the ordered sequence of words between positions $i$ and $j$ relative to the target word. 
%
%% 	\item Part-Of-Speech (POS) tags of surrounding words\\
%% 	The POS tags of the three words to the left and right of the target word are used for disambiguation. If a word in the window is not in the same sentence, its POS tag will be assigned as null. %The default POS tagger in the OpenNLP toolkit~\footnote{\url{http://opennlp.apache.org/}} is used.
%% \end{itemize} 
%
%Each of the features are binary features, and IMS trains a model for each word. IMS then uses an SVM for classification. IMS is open-source, provides state-of-the-art performance at the time of its publication, and is easy to extend. As such, our proposed approach focuses heavily on IMS. 
%
%Training data is required to train IMS, which is a supervised system. 
%An example of training data for training WSD system is the One-Million Sense-Tagged Instances \cite{taghipour2015one}. This is the largest dataset we know of for training WSD systems, and we make use of it for training our systems for the All-Words tasks. 
%
%WSD systems can be evaluated using either fine-grained scoring or coarse-grained scoring. In fine-grained scoring, every sense is equally distinct from each other, and answers must exactly match. In coarse-grained scoring, similar senses are grouped and treated as a single sense. A main bottleneck to Word Sense Disambiguation is the granularity of senses. Since word senses are subjective, and the boundaries between each sense is not always well-defined, an important measure for any task is the inter-annotator agreement. The inter-annotator agreement is considered the upper bound of a task. 
%
%A problem of Word Sense Disambiguation is that the granularity of senses are subjective and may not be well-defined. WordNet is a fine-grained resource, and even human annotators have trouble distinguishing between different senses of a word \cite{edmonds2002introduction}. 
%%In some WSD tasks during Senseval, coarse-grained scoring was done in order to deal with this problem. In these evaluations, similar senses of a word are clustered together and are considered to be the same sense. 
%
%Cross-Lingual WSD was partially conceived as a further attempt to solve this issue. In Cross-Lingual WSD, the specificity of a sense is determined by its correct translation. The sense inventory is the possible translations of each word in another language. Two instances are said to have the same sense if they map to the same translation in that language. In SemEval-2010~\footnote{\url{http://stel.ub.edu/semeval2010-coref/}}, a task for Cross-Lingual WSD was introduced. SemEval-2013~\footnote{\url{https://www.cs.york.ac.uk/semeval-2013/}} featured the second iteration of this task. These tasks were tasks in which an English noun were the targeted words, and the word senses were the translations in Dutch, French, Italian, Spanish and German. 
%
%
%Traditional WSD approaches are used in Cross-Lingual WSD, although some approaches make use of Statistical Machine Translation methods and features from translation. Cross-Lingual WSD involves training by making use of parallel or multilingual corpora. In the Cross-Lingual WSD task in SemEval-2013, the top approaches used a classification approach or a statistical machine translation approach. 
%
%In NLP, words can be represented in a vector space model. Traditionally, this has been done with {\it one-hot} binary vectors, where there is only one non-zero value in a high-dimensional vector. In this encoding, each dimension represents the presence of a word, and the number of dimensions of the vector space is the size of the vocabulary. In one-hot encoding, all words are considered to be independent of each other. A problem with one-hot encoding is that the large number of dimensions makes machine learning vulnerable to over-fitting. There is no notion of word similarity and all words are independent of each other. A distributed representation of words, such as word embeddings, resolves these problems by encoding words into a low dimensional space. In word embeddings, information about a word is distributed across multiple dimensions, and similar words are expected to be close to each other. Examples of word embeddings are Continuous Bag of Words \cite{mikolovword2vec}, Collobert \& Weston's Embeddings \cite{collobert2008unified}, and GLoVe \cite{pennington2014glove}. We implemented and evaluated the use of word embedding features using these embeddings in IMS. 
%
%
%An unsupervised approach using word embeddings for WSD is described by Chen \shortcite{chen2014}. This uses a model for finding representation of senses, rather than just for words, initialised using WordNet's glosses of senses. These sense vectors can then be used during Word Sense Disambiguation. A context vector can be computed by taking the average of the words in a sentence. For disambiguating a single word, the sense with the sense vector that gives maximum Cosine Similarity with this context vector is chosen as the result for disambiguation. Chen {\it et al.} gives an algorithm to disambiguate words starting from the words with fewer senses first. 
%
%A different approach is to work on extending existing WSD systems. Turian \shortcite{Turian10wordrepresentations} suggests that for any existing supervised NLP system, a general way of improving accuracy would be to use unsupervised word representations as additional features. Taghipour \shortcite{Taghipour15} used C\&W embeddings as a starting point and implemented word embeddings as a feature type in IMS. For a specified window, vectors for the surrounding words in the windows, excluding the target word, are obtained from the embeddings and are concatenated, producing $d * (w-1)$ features, where $d$ is the number of dimensions of the vector, and w is the window size. Each feature is a floating point number, which is the value of the vector in a dimension. We note that \cite{Taghipour15} only reported results for C\&W embeddings, and did not experiment on other types of word embeddings.  
%
%Other supervised approaches using word embeddings include AutoExtend \cite{rothe2015autoextend}, which extended word embeddings to create embeddings for synsets and lexemes. In their work, they also extended IMS, but used their own embeddings. Three feature types were introduced by this work, which has some similarities to how Taghipour used word embeddings, but without Taghipour's method of scaling each dimension of the word embeddings. \\
%
%
%% Apart from reviewing work on WSD, we can generalise WSD as a classification problem and look at other approaches to perform classification. We therefore experiment with the approach of using a Neural Network for classification. In Natural Language Processing, much work has been done with Recursive Neural Networks, such as Recurrent Neural Networks, and Recursive Autoencoders. These networks have shown extremely promising results in many NLP classification tasks, such as Sentiment Classification, obtaining state-of-the-art results. 
%
%The structure of this paper will be as follows: we have first reviewed related work and methods regarding WSD, next we will discuss the applications of WSD to a category of educational applications, and outlining possible future work finally in the conclusion. 




\section{Related Work}
% Min: don't need this header -- obvious
% \subsection{Word Sense Disambiguation}

Word Sense Disambiguation is a well-studied problem, in which many
methods have been applied. Existing methods can be broadly categorised
into supervised approaches, where machine learning techniques are used
to learn from labeled training data; and unsupervised knowledge-rich
techniques, which do not rely on labeled data. Unsupervised techniques
are knowledge-rich, and rely heavily on knowledge bases and thesaurus,
such as WordNet. It is noted by Navigli \shortcite{Navigli09wordsense}
that supervised approaches using memory-based learning and SVM
approaches have worked best.
%For these approaches, it is common that the only knowledge used is the first sense in WordNet, which is used as a fallback if the system is unable to disambiguate the word in the test data. 

Supervised approaches involve the extraction of features and then
classification using machine learning. Zhong and Ng
\shortcite{Zhong2010} developed an open-source WSD system, {\it
  ItMakesSense} (hereafter, IMS), which was considered the
state-of-the-art at the time it was developed.  It is a supervised WSD
system, which had to be trained prior to use. IMS uses three feature
types, 1. individual words in the context surrounding the target word,
2. specific ordered sequences of words appearing at specified offsets
from the target word, 3. Part-Of-Speech tags of the surrounding 3
words.

Each of the features are binary features, and IMS trains a model for
each word. IMS then uses an support vector machine (SVM) for
classification. IMS is open-source, provides state-of-the-art
performance, and is easy to extend. As such, our work features IMS and
extends off of this backbone.

Training data is required to train IMS.  We make use of the
One-Million Sense-Tagged Instances \cite{taghipour2015one} dataset,
which is the largest dataset we know of for training WSD systems, in
training our systems for the All Words tasks.

WSD systems can be evaluated using either fine-grained scoring or
coarse-grained scoring. Under fine-grained scoring, every sense is
equally distinct from each other, and answers must exactly match. 
% Min: need to introduce wordnet somewhere.
% Hong Jin: added context that wordnet is the sense inventory
WordNet is often used as the sense inventory for monolingual WSD tasks. However, WordNet is a fine-grained resource, and even human annotators have
trouble distinguishing between different senses of a word
\cite{edmonds2002introduction}.  In contrast, under coarse-grained
scoring, similar senses are grouped and treated as a single sense.  In
some WSD tasks in SemEval, coarse-grained scoring was done in order
to deal with the problem of reliably distinguishing fine-grained
senses. 
% Hong Jin: removed because repetitive
%We note that WSD is somewhat subjective, in that the grouping
%and granularity of senses are debatable and may not be well-defined.
%% Min: not yet clear where this goes
% An important measure for any classification task is the
% inter-annotator agreement. The inter-annotator agreement is considered
% the upper bound of a task.

\subsection{Cross-Lingual Word Sense Disambiguation}
Cross-Lingual WSD was, in part, conceived as a further attempt to
solve this issue. In Cross-Lingual WSD, the specificity of a sense is
determined by its correct translation in another language. The sense
inventory is the possible translations of each word in another
language. Two instances are said to have the same sense if they map to
the same translation in that language.
SemEval-2010~\cite{Lefever2010}\footnote{\url{http://stel.ub.edu/semeval2010-coref/}}
and SemEval-2013~\cite{Lefever2013}\footnote{\url{https://www.cs.york.ac.uk/semeval-2013/}}
featured iterations of this task. These tasks featured
English nouns as the source words, and word senses as translations in
Dutch, French, Italian, Spanish and German.

Traditional WSD approaches are used in Cross-Lingual WSD, although
some approaches leverage statistical machine translation (SMT) methods
and features from translation. Cross-Lingual WSD involves training by
making use of parallel or multilingual corpora. In the Cross-Lingual
WSD task in SemEval-2013, the top performing approaches used either
classification or SMT approaches.

\subsection{WSD with Word Embeddings}

In NLP, words can be represented 
%in a vector space
%model. 
%Traditionally, this has been done with {\it one-hot} binary
%vectors, where there is only one non-zero value in a high-dimensional
%vector. In this encoding, each dimension represents the presence of a
%word, and the number of dimensions of the vector space is the size of
%the vocabulary. In one-hot encoding, all words are considered to be
%independent of each other; there is no notion of word similarity and
%all words are independent of each other.  A key weakness with the
%one-hot representation is that the large number of dimensions makes
%machine learning prone to overfitting.  
with a distributed representation, such as word embeddings, which encodes
words into a low dimensional space. In word embeddings, information
about a word is distributed across multiple dimensions, and similar
words are expected to be close to each other in the vector space. Examples of word
embeddings are Continuous Bag of Words \cite{mikolovword2vec},
Collobert \& Weston's Embeddings \cite{collobert2008unified}, and
GLoVe \cite{pennington2014glove}. We implemented and evaluated the use
of word embedding features using these three embeddings in IMS.

% An unsupervised approach using word embeddings for WSD is described by
% Chen \shortcite{chen2014}. This uses a model for finding
% representation of senses, rather than just for words, initialized
% using WordNet's glosses of senses. These sense vectors can then be
% used during Word Sense Disambiguation. A context vector can be
% computed by taking the average of the words in a sentence. For
% disambiguating a single word, the sense with the sense vector that
% gives maximum Cosine Similarity with this context vector is chosen as
% the result for disambiguation. Chen {\it et al.} gives an algorithm to
% disambiguate words starting from the words with fewer senses first.

An unsupervised approach using word embeddings for WSD is described by Chen \shortcite{chen2014}. This approach finds representation of senses, instead of words, and computes a context vector which is used during disambiguation. 

A different approach is to work on extending existing WSD
systems. Turian \shortcite{Turian10wordrepresentations} suggests that
for any existing supervised NLP system, a general way of improving
accuracy would be to use unsupervised word representations as
additional features. Taghipour \shortcite{Taghipour15} used C\&W
embeddings as a starting point and implemented word embeddings as a
feature type in IMS. For a specified window, vectors for the
surrounding words in the windows, excluding the target word, are
obtained from the embeddings and are concatenated, producing $d *
(w-1)$ features, where $d$ is the number of dimensions of the vector,
and $w$ is the window size. Each feature is a floating point number,
which is the value of the vector in a dimension. We note that
\cite{Taghipour15} only reported results for C\&W embeddings, and did
not experiment on other types of word embeddings.

Other supervised approaches using word embeddings include AutoExtend
\cite{rothe2015autoextend}, which extended word embeddings to create
embeddings for synsets and lexemes. In their work, they also extended
IMS, but used their own embeddings. The feature types
introduced by this work bear similarities to how Taghipour used
word embeddings, but without Taghipour's method of scaling each
dimension of the word embeddings.

% Min: BUG fix this into prose if needed. 
% Hong Jin: not needed
% Summary:
%  - Not many comparison works: \cite{Iacobacci2016}

To conclude, word embeddings have been used in several methods to
improve on state-of-the-art results in WSD. However, to date, there
has been little work investigating how different word embeddings and
parameters affect performance of baseline methods of WSD. As far as we
know, there has only been one paper comparing the different word
embeddings with the use of basic composition methods in WSD. Iacobacci
\shortcite{Iacobacci2016} performed an evaluation study of different
parameters when enhancing an existing supervised WSD system with word
embeddings. Iacobacci noted that the integration of Word2Vec
% Min: skip grams are different from the CBOW
(skip-gram) with IMS was consistently helpful and provided the best
performance. Iacobacci also noted that the composition methods of
average and concatenation produced small gains relative to the other
composition strategies introduced. However, Iacobacci did not
investigate the use of \cite{Taghipour15}'s scaling strategy, which
was crucial to improve the performance of IMS.

We also did not find any recent work attempting to integrate modern
WSD systems for real-world education usage, and to evaluate the WSD
system based on the requirements and suitability for education use.
We aim to fill this gap in applied WSD with this work.

	\section{Methods}
\label{section:methods}
% Tao: Which classifier did you use?
% Hong Jin: SVM?

As Navigli \shortcite{Navigli09wordsense} noted that supervised approaches have performed best in WSD, we focus on integrating word embeddings in supervised approaches.
%We explored two approaches for performing supervised WSD using word embeddings. Firstly, we experimented with different methods of composing word embeddings to represent the context of a word and used this in conjunction with IMS, a state-of-the-art supervised WSD system. 
%Secondly, we explored the use of Neural Networks, which have produced state-of-the-art performance in many NLP tasks, in performing WSD. A LSTM network, a type of Recurrent Neural Network, is evaluated. 
We explore the use of word embeddings in IMS. We focused our work on Continuous Bag of Words (CBOW) from Word2Vec,  Global Vectors for Word Representation (GLoVe) and Collobert \& Weston's Embeddings. The CBOW embeddings were trained over Wikipedia, while the publicly available vectors from GloVe and C\&W were used. Word2Vec provides 2 architectures for learning word embeddings, Skip-gram and CBOW. In contrast to Iacobacci \shortcite{Iacobacci2016} which focused on Skip-gram, we focused our work on CBOW.
In our first set of evaluation, we used tasks from Senseval-2, Senseval-3 and SemEval-2007 to evaluate the performance of our classifiers on monolingual WSD. We do this to first validate that our approach is a sound approach of performing WSD, showing improved or identical scores to state-of-the-art systems in most tasks. 

Similar to the work by Taghipour \shortcite{Taghipour15}, we experimented with the use of word embeddings as feature types in IMS. However, we did not just experiment using C\&W embeddings, as different word embeddings are known to vary in quality when evaluated on different tasks \cite{schnabel2015evaluation}. We performed evaluation on several tasks. For the Lexical Sample (LS) tasks of Senseval-2 \cite{senseval2-LS-kilgarriff2001} and Senseval-3 \cite{senseval3-LS-mihalcea2004}, we evaluated our system using fine-grained scoring. For the All Words (AW) tasks, fine-grained scoring is done for Senseval-2 \cite{senseval2-AW-palmer2001} and Senseval-3 \cite{senseval3-AW-snyder2004}, and both the fine \cite{semeval2007-fine-pradhan2007} and coarse-grained \cite{semeval2007-coarse-navigli2007} All Words tasks in SemEval-2007 were used. In order to evaluate our features on the All Words task, we trained IMS and the different combinations of features on the One Million Sense-Tagged corpus \cite{taghipour2015one}.
To compose word vectors, one baseline method is to sum up the word vectors of the words in the surrounding context or sentence. We primarily experimented on this method of composition, due to its good performance and short training time. For this, every word vector for every lemma in the sentence, excluding the target word, was summed into a context vector, resulting in $d$ features. Stopwords and punctuation are discarded. In Turian's \shortcite{Turian10wordrepresentations} work, two hyperparameters, the capacity (number of dimensions) and size of the word embeddings, were tuned in his experiments. We did the same in our experiments.

As the other features in IMS are binary features, we need to scale down the word embeddings, as suggested by Turian \shortcite{Turian10wordrepresentations}. This is because the range of the word embeddings are not bounded, and therefore can have more influence than binary features. The embeddings are scaled to control their standard deviations. We implemented a variant of this technique as done by Taghipour \shortcite{Taghipour15}, in which we set the target standard deviation for each dimension. A comparison of different values of the scaling parameter, $\sigma$ is done. For each $i \in \{1, 2, .. d\}$:
\\

$E_{i} \leftarrow \sigma \times \frac{E_{i}}{stdev(E_{i})} $, where $\sigma$ is a scaling constant that sets the target standard deviation
\\ 

We evaluate the effect of varying the scaling factor with the feature of the sum of the surrounding word vectors. Word embeddings of 50 dimensions were used.


%\singlespacing

\begin{table}[ht]
	\caption{Effects of varying scaling factor on C\&W embeddings }
	\label{table:wordembeddings-accuracy}
	\begin{center}
		\begin{tabular}{| p{7cm} | p{4cm} | p{4cm} |}
			\hline
			Method & Senseval-2 Accuracy & Senseval-3 Accuracy \\
			\hline
			C\&W, unscaled & 0.569 & 0.641 \\
			\hline
			C\&W, $\sigma _{=0.15}$ & 0.665 & 0.731 \\
			\hline
			C\&W, $\sigma _{=0.1}$ & {\bf0.672} & {\bf0.739} \\
			\hline
			C\&W, $\sigma _{=0.05}$ & 0.664 & 0.735 \\
			\hline
			
		\end{tabular}
	\end{center}
\end{table}
We experimented with different word embeddings and varied the value of $\sigma$. Like Turian \shortcite{Turian10wordrepresentations} and Taghipour \shortcite{Taghipour15}, we found that a value of 0.1 for $\sigma$ worked well, seen in Table \ref{table:wordembeddings-accuracy}. The number of dimensions, known as the capacity, of the word embeddings was tuned by Turian \shortcite{Turian10wordrepresentations}. Hence, we varied the values of the capacity for CBOW and GloVe. As we can see in Tables \ref{table:wordembeddings-word2vec-accuracy} and \ref{table:wordembeddings-glove-accuracy}, in CBOW and GloVe, the context sum feature works optimally with 50 dimensions. 

In Table \ref{table:top-LS}, we compare the performances of our system on Senseval-2 (held in 2001) and Senseval-3's (held in 2004) Lexical Sample tasks with IMS, the top system for each task, and other recent systems that evaluated on the same task.


\begin{table}[ht]
	\caption{Effects of varying capacity on CBOW}
	\label{table:wordembeddings-word2vec-accuracy}
	\begin{center}
		\begin{tabular}{| p{7cm} | p{4cm} | p{4cm} |}
			\hline
			Method & Senseval-2 Accuracy & Senseval-3 Accuracy \\
			\hline
			CBOW, $dimensions_{=50}$ & {\bf0.680} & {\bf0.741} \\
			\hline
			CBOW, $dimensions_{=300}$ & 0.669 & 0.731 \\
			\hline
			
		\end{tabular}
	\end{center}
\end{table}

\begin{table}[ht]
	\caption{Effects of varying capacity on GloVe}
	\label{table:wordembeddings-glove-accuracy}
	\begin{center}
		\begin{tabular}{| p{7cm} | p{4cm} | p{4cm} |}
			\hline
			Method & Senseval-2 Accuracy & Senseval-3 Accuracy \\
			\hline
			GloVe, $dimensions_{=50}$ & {\bf0.678} & {\bf0.741} \\
			\hline
			GloVe, $dimensions_{=100}$ & 0.668 & 0.734 \\
			\hline
			GloVe, $dimensions_{=200}$ & 0.666 & 0.73 \\
			\hline
			
		\end{tabular}
	\end{center}
\end{table}



Because each dimension is a feature that is used by IMS, if there are more dimensions, then there are more features. This may result in over-fitting on small datasets. This is a possible reason that a smaller number of dimensions work better. 

\begin{table}
	\caption{Comparison of systems on Lexical Sample tasks. Rank 1 system refers to the top system for the specified task during the evaluation}
	\label{table:top-LS}
	\begin{center}
		\begin{tabular}{| p{7cm} | p{4cm} | p{4cm} |}
			\hline
			Method & Senseval-2 LS Accuracy & Senseval-3 LS Accuracy \\
			\hline
			IMS + Context Sum, CBOW $\sigma _{=0.1}$ (proposal) & 0.680 & 0.741 \\
			\hline
            IMS + CBOW $\sigma _{=0.15}$ (proposal) & 0.670 & 0.734 \\
			\hline
			
			IMS & 0.653 & 0.726\\
			\hline
			Rank 1 System & 0.642 & 0.729 \\
			\hline
			\newcite{rothe2015autoextend} & 0.666 & 0.736 \\
			\hline
			\newcite{Taghipour15} & 0.662 & 0.734 \\
			\hline
           	IMS + Word2Vec (Skip-gram) \shortcite{Iacobacci2016}  & {\bf0.699} & {\bf0.752} \\
			\hline
			Most Frequent Sense (Baseline) & 0.476 & 0.552 \\
			\hline
		\end{tabular}
	\end{center}
\end{table}



Apart from the Lexical Sample tasks, we evaluated our systems on the All Words tasks of Senseval-2, Senseval-3 and SemEval-2007, which were evaluated on by Zhong \shortcite{Zhong2010}. As seen in Table \ref{table:All-AW}, our enhancements to IMS to make use of word embeddings give better results on All Words task than the original IMS and the respective Rank 1 Systems during the task. It also outperforms several recent systems developed and evaluated in recent papers. We note that although our system increased accuracy on IMS on several
% Tao: which test did you use? 
% Hong Jin: mcnemar's test
tasks, the differences were not statistically significant.

\begin{table}
	\caption{Accuracy of our system on Senseval-2, Senseval-3, SemEval-2007 All Words task}
	\label{table:All-AW}
	\begin{center}
		\begin{tabular}{| p{4cm} | p{2cm} | p{3cm} | p{3cm} | p{3cm} | }
			\hline
			Method & Senseval-2 AW Accuracy & Senseval-3 AW Accuracy & SemEval-2007 Fine-Grained & SemEval-2007 Coarse-grained \\
			\hline
			IMS + CBOW, 
			
			$\sigma _{=0.1}$ (proposal) & 0.677 & 0.679 & 0.604 & 0.826  \\
			\hline
            IMS + 
			
			CBOW, $\sigma _{=0.15}$ (proposal) & 0.673 & 0.675 & {\bf0.615} & {\bf 0.828 } \\
			\hline
			
			\cite{Taghipour15} & -& {\bf0.682} & - & - \\
			\hline
			\cite{chen2014} & - & - & - & 0.826  \\
			\hline
			IMS (on One Million Sense-Tagged dataset) & 0.682 & 0.674 & 0.585 & 0.816 \\
            \hline
            IMS + Skip-gram \cite{Iacobacci2016}  & {\bf0.683} & {\bf0.682} & 0.591 & - \\
			\hline
			IMS (original) & 0.682 & 0.676 & 0.583 & 0.826   \\
			\hline
			Top System during the task & {\bf0.69} & 0.652 & 0.591 & 0.825  \\
			\hline
			WordNet Sense 1 & 0.619 & 0.624 & 0.514 & 0.789\\
			\hline
		\end{tabular}
	\end{center}
\end{table}


It can be seen in Table \ref{table:top-LS} and \ref{table:All-AW} that the simple enhancement of integrating word embedding using the baseline composition method, followed by the scaling step, improves the existing IMS system, and we get performance comparable to or better than top approaches in both Lexical Sample tasks and All Words tasks. 

\begin{table}
	\caption{Accuracy of adding word embeddings to IMS on Senseval-2, Senseval-3 Lexical Sample tasks and SemEval-2007 All Words task}
	\label{table:full}
	\begin{center}
\begin{tabular}{|p{1cm}|p{0.5cm}|p{1cm}|p{1.5cm}|p{1.5cm}|p{1.5cm}|p{1.5cm}|p{1.5cm}|p{1.5cm}|p{1.5cm}|}
Type & Size & Scaling & Compose & SE-2 LS & SE-3 LS & SE-2 AW & SE-3 AW & SE-2007 Fine grained & SE-2007 Coarse grained \\
\hline
Senna&50&0.05&Sum&0.666&0.734&0.679&0.673&0.594&0.818 \\
\hline
Senna&50&0.1&Sum&0.671&0.738&0.678&0.673&0.6&0.819 \\
\hline
Senna&50&0.15&Sum&0.666&0.732&0.675&0.672&0.598&0.817 \\
\hline
CBOW&50&0.05&Sum&0.672&{\bf 0.744}&{\bf 0.68}&0.677&0.604&0.824\\
\hline
CBOW&50&0.1&Sum&{\bf 0.68}&0.741&0.677&0.679 &0.604 & 0.826\\
\hline
CBOW&50&0.15&Sum&0.67&0.734&0.673&0.675&{\bf 0.615}&{\bf 0.828}\\
\hline
Glove&50&0.05&Sum&0.675&0.738&0.676&0.678&0.596&0.819 \\
\hline
Glove&50&0.1&Sum&0.679&0.741&0.678&0.68&0.594&0.819 \\
\hline
Glove&50&0.15&Sum&0.674&0.731&{\bf 0.68}&0.678&0.591&0.819 \\
\hline
CBOW&200&0.05&Sum&0.679&0.742&0.679&0.68&0.602&0.823 \\
\hline
CBOW&200&0.1&Sum&0.669&0.731&0.676&0.675&0.602&0.82 \\
\hline
CBOW&200&0.15&Sum&0.651&0.715&0.667&0.673&0.594&0.822 \\
\hline
Glove&200&0.05&Sum&0.682&0.741&0.68&{\bf0.682}&0.6&0.823 \\
\hline
Glove&200&0.1&Sum&0.666&0.73&0.677&0.679&0.591&0.827 \\
\hline
Glove&200&0.15&Sum&0.654&0.706&0.674&0.675&0.591&0.826 \\
\hline
Senna&50&0.1&Concat&0.659&0.724&0.679&0.674&0.585&0.818 \\
\hline
CBOW&50&0.1&Concat&0.66&0.725&0.678&0.672&0.581&0.816\\
\hline
CBOW&200&0.1&Concat&0.667&0.729&0.675&0.67&0.591&0.819\\
\hline
Glove&50&0.1&Concat&0.657&0.722&0.679&0.671&0.583&0.818\\
\hline
Glove&200&0.1&Concat&0.664&0.728&0.677&0.669&0.587&0.817\\
\hline
\end{tabular}
	\end{center}
\end{table}

We note that a smaller size of the word embeddings generally improved performance on the Lexical Sample task, however, this effect was not observed in the All Words task. We also note that relatively poorer performance in the Lexical Sample tasks may not necessarily result in poor performance on the All Words task. We see from the results that the combination of \cite{Taghipour15}'s scaling strategy and summation produced results better then \cite{Iacobacci2016}'s concatenation and average combination strategy (0.651 and 0.654), suggesting that the scaling factor is important for the integration of word embeddings for supervised WSD. 


%\iffalse
\subsection{LSTM Network}

A Long Short Term Memory (LSTM) network is a type of Recurrent Neural Network, that has been shown to have good performance on many NLP classification tasks. Unlike normal Neural Networks which can only accept a fixed size vector as an input, Recurrent neural networks accept variable sized inputs. As such, recurrent neural networks can operate over sequences of word vectors and perform operations on them sequentially. The potential benefit of this approach over our existing approach in IMS is this is that the neural network is able to use information about the sequence of words in classification. Examples of using a neural network for classification are \cite{socher2011parsing}, and \cite{socher2013recursive}. Yuan \shortcite{yuan2016word} explores the use of LSTM networks, using label propagation, for WSD. In our approach, we explore a simpler naive approach instead without the use of label propagation.


For the Lexical Sample tasks, we train the model on the training data provided for the task. For the All Words task, we trained the model on the One Million Sense-Tagged dataset. For each task, similar to IMS, we train a model for each word. 

One problem we encountered with this approach was that there is very few training examples per sense. Although as a whole, the dataset had many training examples, we trained a model for each word, resulting in a fairly small number of relevant training instances for a single word. 





\begin{table}
	\caption{Accuracy of our Neural Network approach on the Lexical Sample tasks}
	\label{table:NN-LS}
	\begin{center}
		\begin{tabular}{| p{6cm} | p{4cm} | p{4cm} |}
			\hline
			Method & Senseval-2 Accuracy & Senseval-3 Accuracy \\
			\hline
			LSTM approach (Proposed) & 0.458  & 0.603 \\
			
			\hline
			IMS & 0.653 & {\bf0.726}\\
			\hline
			Rank 1 System during the task & {\bf0.642} & 0.729 \\
			\hline
			Most Frequent Sense (Baseline) & 0.476 & 0.552 \\
			\hline
		\end{tabular}
	\end{center}
\end{table}

\begin{table}
	\caption{Accuracy of our Neural Network approach on the All Words tasks}
	\label{table:NN_AW}
	\begin{center}
		\begin{tabular}{| p{7cm} | p{2cm} | p{2cm} | p{2cm} | }
			\hline
			Method & Senseval-2 Accuracy & Senseval-3 Accuracy\\
			\hline
			LSTM approach (Proposed) & 0.619  & 0.623  \\
			
			\hline
			IMS (trained on One Million Sense-Tagged dataset) & 0.682 & {\bf0.674} \\
			\hline
			Rank 1 System during the task & {\bf0.69} & 0.652  \\
			\hline
			Wordnet Sense 1 & 0.619 & 0.624  \\
			\hline
		\end{tabular}
	\end{center}
\end{table}

The performance of the neural network is poor in every task, as we can see in both Tables \ref{table:NN-LS} and \ref{table:NN_AW}. The models converge to just using the most common sense. The reason for this is that WSD suffers from the problem of data sparsity. Although there are many training instances in total, the average amount of training examples for each individual sense is low. 
\section{English-Chinese Cross-Lingual Word Sense Disambiguation}
\label{section:CLWSD}

% Tao: papers about CLWSD

% sem-eval clwsd task: \cite{Lefever2010}, \cite{Lefever2013}
% cross-lingual embeddings: \cite{Shi2015}, \cite{Coulmance2015}, \cite{Aldarmaki2016}
% build word embeddings for WSD: \cite{Guo2014}
% clwsd dataset: \cite{Rekabsaz2016} (English-Persian)

% survey on wsd: \cite{Navigli2009}


We now evaluate our proposal on the Cross-Lingual Word Sense
Disambiguation task.  One key application of such task is to
facilitate language learning systems.  For example, {\it
 MindTheWord}\footnote{\url{https://chrome.google.com/webstore/detail/mindtheword/fabjlaokbhaoehejcoblhahcekmogbom}}
and {\it WordNews}~\cite{tao2014} are two applications that allow
users to learn vocabulary of a second language in context, in the form
of providing translations of words in an online article.
%which is required for effective acquisition of vocabulary \cite{Hirsch03readingcomprehension}. These applications often rely on translation systems to provide translations. 
In this work, we model this problem of finding translations of words
as a variant of WSD, Cross-Lingual Word Sense Disambiguation, as
formalized in \cite{tao2014}.
% Tao: stop here. 

In the previous section, we have validated and compared enhancements to
IMS using word embeddings. These have produced results comparable to,
and in some cases, better than state-of-the-art performance on the
monolingual WSD tasks. We further evaluate our approach for use in the
Cross-Lingual Word Sense Disambiguation for performing contextually
appropriate translations of single words. To accomplish this, we first
construct a English--Chinese Cross-Lingual WSD dataset. For our sense
inventory, we work with the existing dictionary in the open-source
educational application, WordNews \cite{tao2014}, which contains a
dictionary of English words and their possible Chinese
translations. We finally deploy the trained system as a fork of the
original WordNews.

\subsection{Dataset}
In order to construct an evaluation dataset, we hired human annotators and constructed a English--
Chinese Cross-Lingual WSD dataset using sentences from recent news articles. As far as we know, there 
is no existing publicly available English--Chinese Cross-Lingual WSD dataset. As the dataset is 
constructed using recent news data, it is a good representation for the use case in WordNews.{\footnote{The dataset can be obtained at %{\url{  https://kanghj.github.io/eng_chinese_news_clwsd_dataset/}}}}
{\url{https://to.be.made.public/if/accepted}}}

To obtain the gold standard for this data set, we hired 18 annotators to select the right translations for a given word and its context. There are 697 instances in total in our dataset, with a total of 251 target words to disambiguate, that were each multiply-annotated by 3 different annotators. Each annotator disambiguated 110+ instances (15 annotators with 116 instances, 3 with 117) in hard-copy. The annotators are all bilingual undergraduate students, who are native Chinese speakers. 

For each instance, which contains a single English target word to disambiguate, we include the sentence it appears in and its adjacent sentences as its context. Each instance contains possible translations of the word. 
% Hong Jin: Omitted since it mentioned later 
%The sense inventory is from a dictionary of English--Chinese pairs, crawled from Google Translate and Bing Translator.
The annotators selected all Chinese words that had an identical meaning to the English target word. If the word cannot be appropriately translated, we instructed annotators to leave the annotation blank. The annotators provided their own translations if they believe that there is a suitable translation, but which was not provided by the crawled dictionary. 

%In WSD, it is important to obtain the inter-annotator agreement of the dataset. 
The concept of a sense is a human construct, and therefore, as earlier elaborated on when discussing sense granularity, it is %subjective and 
may be difficult for human annotators to agree on the correct answer. 
%We try to measure the inter-annotator agreement using pairwise Cohen's Kappa. 
Our annotation task differs from the usual since we allow users to select multiple labels and can also add new labels to each case if they do not agree with any label provided. As such, applying the Cohen's Kappa as it is for measuring the inter-annotator agreement as it is does not work for our annotated dataset. %We note that some work has been done for multi-label Kappa, such as by Rosenberg \shortcite{rosenberg2004augmenting}, however the situation described  is different from our case, as we cannot assume a uniform distribution of labels%or that there is a primary label among the multiple labels selected by an annotator.
We are also unable to compute the probably of chance agreement by word, since there are few test instances per word in our dataset.

The Kappa equation is given as 
$\kappa = \frac{p_A - p_E}{1 - p_E} $.
To compute $p_A$ for $\kappa$, we use a simplified, optimistic approach where we select one annotated label out of possibly multiple selected labels for each annotator. We always choose the label that results in an agreement between the pair, if such a label exist. For $p_E$ (the probability of chance agreement), as the labels of each case are different, we consider the labels in terms of how frequent they occur in the training data. 
We only consider the top 3 most frequent senses for each word %, and ignore the other labels 
due to the skewness of the sense distribution. 
We first compute the probability of an annotator selecting each of the top three frequent senses, $p_E$ is then equals to the sum of the probability that both annotators selected one of the three top senses by chance. 

%We present the probabilities that an annotator will select each of the top three senses in Table \ref{table:IAA}.
%The value of $p_E$ by this proposed method of computation is 0.186.
The pairwise value by this proposed method of $\kappa$ is obtained is 0.42. We interpreted this as a moderate level of agreement. We note that there is a large number of possible labels for each case, which is known to affect the value of $\kappa$ negatively. This is exacerbated as we allow the annotators to add new labels. 

% \begin{table}[ht]
% 	\caption{Probability of an annotator annotating the top three senses.}
% 	\label{table:IAA}
% 	\begin{center}
% 		\begin{tabular}{| p{4cm} | p{4cm}  | p{4cm} | }
% 			\hline
% 			Most Frequent Sense & 2nd Most Frequent Sense & 3rd Most Frequent Senses\\
% 			\hline
% 			0.343 & 0.206 & 0.161\\						
% 			\hline
% 		\end{tabular}
% 	\end{center}
% \end{table}

%As we consider any overlap in annotated labels to be a match, this approach may overestimate the agreement between annotators. %However, in our dataset, a significant number of annotators (5 out of 18) only selected a single translation in the dataset instead of every suitable translation. 
%However 
In this annotation task, as we consider the possible translations as  fine-grained, the value of agreement is likely to be underestimated in this case. Hence, we believe that clustering of similar translations during annotation is required in order to deal with the issue of sense granularity in Cross-Lingual WSD. 
To overcome this problem, we used different configurations of granularity during evaluation of our system. 
% Hong Jin : omited this explanation since it may be clear enough from the table
%In the most relaxed configuration, we assume that all annotations by the annotators are correct answers. Under our strictest configuration, all three annotators must agree on the translation before it is considered to be the correct sense. 
For all configurations,
we remove instances from the dataset if it does not have a correct sense. 

% Hong Jin : moved this to table header instead
%We excluded instances with out-of-vocabulary annotations (added by the annotators if they did not think any of the provided translations are suitable) were excluded from the test set.

% Hong Jin: omitted since the table duplicates this information

%For the first configuration, we included all instances annotated by the participants. For the second configuration, we omitted bad instances and only consider a translation to be correct if more than one participant agreed on that translation. For the third configuration, we included only answers where all three participants agreed on the answer. 
We also noticed that some target words were part of a proper noun, such as the word 'white' in 'White House'. This led to some confusion among annotators, so we omitted instances where the target word is part of a proper noun. Statistics of the test dataset after filtering out different cases are given in Table \ref{table:CLWSD-test-stats-no-ne}.

\begin{table}[ht]
	\caption{Statistics of our new annotated Chinese-English crosslingual WSD dataset. Out-of-vocabulary (OOV) annotations refer to annotations added by the annotators}
    %% Muthu: if this is the our new annotated dataset we need to make it clear in the caption because this is a contribution of the paper.
	\label{table:CLWSD-test-stats-no-ne}
	\begin{center}
		\begin{tabular}{| p{8cm} | r| r|}
			\hline
			{\bf Configuration} & {\bf \# of instances} & {\bf \# of unique target words} \\
			\hline
			Include all & 653 & 251\\ 
			\hline
			Exclude instances with OOV annotations & 481 & 206 \\						
			\hline
			Exclude instances without at least partial agreement & 412 & 193 \\
			\hline
			Exclude instances without complete agreement & 229 & 136 \\
			\hline
		\end{tabular}
	\end{center}
\end{table}

\subsection{Experiments}

As previously described, IMS is a supervised system requiring training data before use. We constructed data by processing a parallel corpus, the news section of the UM-Corpus \cite{tian2014corpus}, and performing word alignment. We used the dictionary provided by \cite{tao2014} as the sense inventory, which we further expanded using translations from Bing Translator and Google Translate. For construction of the training dataset, word alignment is used to assign Chinese words as training labels for each English target word. GIZA++ \cite {och03} is used for word alignment. To evaluate our system, we compare the results of the method described in \cite{tao2014}, which uses Bing Translator and word alignment to obtain translations. We use the configuration where every annotation is considered to be correct for our main evaluation since this is closer to a coarse-grained evaluation. 

\begin{table}[ht]
	\caption{Results of our systems on the Cross-Lingual WSD dataset, excluding named entities. Instances with out-of-vocabulary annotations are removed. All annotations are considered correct answers.}
	\label{table:CLWSD-test-results}
	\begin{center}

			\begin{tabular}{| p{9cm}| r| }
				\hline
				\textbf{Method} & \textbf{Accuracy} \\
				\hline
				Bing Translator + 
                word alignment (baseline) & 0.559  \\
				\hline
				IMS & 0.752  \\
				\hline
                IMS + CBOW, 50 dimensions, $\sigma _{=0.05}$ (proposed) &  0.763  \\
				\hline
				IMS + CBOW, 50 dimensions, $\sigma _{=0.1}$ (proposed) & {\bf 0.772}  \\                                
                \hline
                IMS + CBOW, 50 dimensions, $\sigma _{=0.15}$ (proposed) & 0.767  \\
                \hline
                % Omitting the single GloVe experiment, did not have time to complete the other configurations
        %        IMS + GLoVe, 50 dimensions, $\sigma _{=0.1}$ (proposed) & 0.761  \\


		%		\hline
			\end{tabular}

	\end{center}
\end{table}



It can be seen that word embeddings improves the performance on Cross-Lingual WSD. Similar to our observations for monolingual WSD, the use of both CBOW and GLoVe improved performance. However, the improvements from the word embeddings feature type over IMS was not statistically significant at 95\% confidence level. This is attributed to the small size of the dataset. 

\subsection{Bing Translator results}
We wish to highlight and explain the poor performance of Bing Translator with our annotated dataset as seen in Table \ref{table:CLWSD-test-results}. This could be because Bing Translator performs translation at the phrase level. Therefore, many of the target words are not translated individually and are translated only as part of a larger unit, making it less suitable for the use case in WordNews where only the translation of single words matter. 
% Min: need to put actual (bilingual) example to concretize the illustration?
% Hong Jin: will do this over the weekend
\begin{CJK*}{UTF8}{gbsn}
For example, when translating the word ``little'' in ``These are serious issues and themes, and sometimes {\bf little} kids aren't ready to process and understand these ideas'', Bing Translator provides a translation of ``这些都是严重的问题和主题,有时{\bf 小孩} 不准备处理和理解这些想法'' but does not give an alignment for the word `little' but instead provides an alignment for the entire multi-word unit ``little kids''. 
\end{CJK*}
%Since the gold standard was produced by annotations before we ran the experiments, none of the %participants would indicate that the translation of ``little kids'' is the translation for %``little''. 
As such, the translation would not match any of the annotations provided by our annotators. This is an appropriate treatment since a user of an educational app requesting specifically a translation for the single word ``little'' should not see the translation of the phrase.

\section{Conclusion}
\label{section:conclusion}

After we have evaluated the performance of the systems on the this
Cross-Lingual WSD dataset, we integrate the top-performing system
using word embeddings and the trained models into a fork of the
WordNews system. We experimented and implemented with different
methods of using word embeddings for supervised WSD. We tried two
approaches, by enhancing an existing WSD system, IMS, and by trying a
neural approach using a simple LSTM.  We evaluated our apporach as
well as various methods in WSD, against initial evaluations on the
existing test data sets from Senseval-2, Senseval-3, SemEval-2007. In
a nutshell, adding any pretrained word embedding as a feature type to
IMS resulted in the system performing competitively or better than the
state-of-the-art systems on many of the tasks. This supports
\cite{Iacobacci2016}'s conclusion that concluded that existing
supervised approaches can be augmented with word embeddings to give
better results.

Our findings also validated Iacobacci et al. \shortcite{Iacobacci2016}'s
findings that Word2Vec gave the best performance. However, we also
note that, other than Word2Vec, other publicly available word
embeddings, Collobert \& Weston's embeddings and GLoVe also
consistently enhanced the performance of IMS using the summation
feature with little effort. Other than on the Lexical Sample tasks,
where smaller word embeddings performed better, we also found that the
number of dimensions did not affect results as much as the scaling
parameter. Unlike Iacobacci et al. \shortcite{Iacobacci2016}, we also
found that a simple composition method using summation already gave
good improvements over the standard WSD features, provided that the
scaling method described in \cite{Taghipour15} was performed.

An additional key contribution of our work was to build a
gold-standard English-Chinese Cross-Lingual WSD dataset constructed
with sentences from real news articles and to evaluate our proposed
word embedding approach under this scenario.  Our compiled dataset was
used as evaluation of the task of translating English words on online
news articles. This dataset is made available publicly.  We observed
that word embeddings also improves the performance of WSD in our
Cross-Lingual WSD setting.

As future work, we will examine how to expand the existing dictionary
with more English words of varying difficulty and include more
possible Chinese translations, as we note that there were several
instances in the Cross-Lingual WSD dataset where the annotators did
not choose an existing translation. 

% Hong Jin: removed this since we did not mention this anywhere else
%An extrinsic evaluation of the
%Cross-Lingual WSD system should be done with users of different
%language learning application in order to validate that the quality of
%translations did indeed improve in real world usage.


\bibliographystyle{acl}
\bibliography{socreport}
\end{document}

\section{Distractors Generation Algorithm}
The key research topic here is to investigate a way to automatically generate suitable distractors for a certain vocabulary test. The distractors are generated in English form.
\subsection{Collecting category-related words}
To generate good category-related distractors, it is essential to gather enough words that are more related in a certain category to serve as distractors candidates. By using the approach discussed in Section 3, we crawled more than 1400 articles for seven categories, with around 200 articles in each category. The confidence factor is selected to be 10, which is suitable to classify enough words into different categories. After this step, there should be sufficient “Category-Related� words in each category.

\subsection{Generating distractors}
The category-related words obtained from the previous step will be used in this step. Out selection strategy in choosing distractors takes following parameters:
\begin{itemize}
\item News website URL
\item News sentence
\item Word to test
\item User’s knowledge level of the word
\end{itemize}
\subsubsection{Detect news category}
After getting the news URL, our system needs to determine the category of the news. Based on the analysis from most popular news URLs, there is a set of common identifiers that can identify the category of the news article. For example, technology news URL often contains “/tech”, “/science”, and if we find these strings in news URL, we will classify this news URL into “Technology” category. The algorithm will go through all category identifier in the list, and will return the category name the moment it finds a match. The current list of category provides reasonable accuracy for the purpose of detecting news category.

\subsubsection{Detect Part-Of-Speech Tag}
Given the target word and the target sentence, it is easy to run the NLTK POS tagger to get the correct POS tag of this word. This step is essential to help select distractors with similar forms, i.e. if the target word is adjective, it will be appropriate to choose three other adjectives, not verbs, as distractors.

\subsubsection{Semantic Distance}
Before we go to explain the next step, it is essential to introduce the semantic distance calculator we used in the server implementation. 

The perspective of semantic relatedness or its inverse, semantic distance, is a concept that indicates the likeness of two words. It is more general than the concept of similarity as stated in WordNet’s synset relation. Similar entities in WordNet are classified into same synset based on their similarity. However, dissimilar entries may also have a close semantic connection by lexical relationships  such as meronymy (car-wheel) and antonymy (hot-cold), or just by any kind of functional relationship or frequent association(pencil-paper, penguin-Antarctica) \cite{ale01}. Semantic distance calculator aims to calculate the semantic relatedness score between two words.

There are many approaches to calculate semantic relatedness score. In this application, we are using Lin Distance \cite{lin98} to calculate the semantic distance between two concepts. The detail of Lin Distance methodology is explained as follows.

Lin attempted to define a measure of semantic similarity that would be both universal and theoretically justified. There are three intuitions that he used as a basis:
\begin{itemize}
\item The similarity between arbitrary objects A and B is related to their commonality; the more commonality they share, the more similar they are;
\item The similarity between A and B is related to the differences between them; the more differences they have, the less similar they are.
\item The maximum similarity between A and B is reached when A and B are identical, no matter how much commonality they share. 
\end{itemize}

Based on the intuition above, Lin proposed his approach in measuring similarity between two concepts c1, c2 in Equation~\ref{equation:Distractor_4}:

\begin{equation}
sim(c1,c2) = \frac{2*log_p(lso(c1,c2))}{log_p(c1)+log_p(c2)}
\label{equation:Distractor_4}
\end{equation}  

where p(c) denotes the probability of encountering concept c, and lso(c1,c2) denotes the lowest common subsumer, which is the lowest node in WordNet hierarchy that is a hypernym of c1 and c2. 

The distance calculator will return a score from 0 to 1, as can be easily seen from the formula above. If the score is closer to 1, it means the two words are closer in semantic sense. This distance calculator will play an important role in the following algorithm. 

\subsubsection{Distractors Selection Algorithm}
Based on the input parameters, at this stage the server has already got the current category of the news article and the correct POS tag of the target word to test. The server is going to generate distractors based on user’s knowledge level of the target word to test.

Knowledge level is 1: This indicates that the user has just learnt this word. The algorithm will randomly select three words from current category’s word list. The reason for using randomization is to avoid the situation that similar distractors are generated every time.

Knowledge level is 2: This indicates that the user has known this word for some times. The algorithm will randomly select two words from the current category’s word list as two distractors. Then the algorithm will randomly select word from the current category’s word list and calculated the semantic distance between the selected word and the target word, once the score is above certain threshold, the selected word will be chose as the third distractor. The selection of threshold value will have a direct effect on the speed of distractors generation process. As a very high threshold value will result in more rounds of calculation in semantic distance calculator, and it will take a long time before the distractors are returned to the front end. After several rounds of analysis of each category’s words and the results returned from semantic distance calculator, the threshold value of 0.1 is selected.

Knowledge level is 3: This indicates that the user has a good understanding of the word already; the algorithm will choose distractors solely based on results returned from semantic distance calculator. Similar to the approach when knowledge level is 2, the algorithm will randomly select word from current category’s word list and calculate the semantic distance between the selected word and the target word. If the score is above certain threshold, the selected word is chosen as one of the distractors. The process is continued until the server can find three distractors. 
